\documentclass[12pt]{article}
\usepackage{fancyhdr}
\usepackage{ushort}
\usepackage{pdfpages}
\usepackage{amsmath, amssymb, amscd}
\usepackage[margin=2.7cm]{geometry}
\usepackage{graphicx}
\usepackage{setspace}
\usepackage{url}
\usepackage{multicol}
\usepackage{hyperref}
\newenvironment{Figure}
  {\par\medskip\noindent\minipage{\linewidth}}
  {\endminipage\par\medskip}

\begin{document}
\thispagestyle{fancy}
\newcommand{\HRule}{\rule{\linewidth}{0.5mm}}
\input{fire_title}
\setlength{\parindent}{0pt}
\tableofcontents
\newpage

\section{Introduction}
This document describes the methodology used to process the fire data and how the processed data, as well as other compositional elements, were used to produce the composition ``Fire''.
\section{Data}
The provided data can be found \href{https://drive.google.com/drive/folders/1sgDxNkrS1AG77fC-4ga5EZ2CEBacdBM3}{here}.
\subsection{Schema}
The schema of the data consists of eight columns: UTC Time, Time, Radiant heat (kW m$^{-2}$), Convective Heat (kW m$^{-2}$), Total Heat (kW m$^{-2}$), Temperature ($^{\circ}$C), Vertical Wind (m s$^{-1}$), Horizontal Wind (m s${^{-1}}$). I will only be using the Total Heat data, which is shown in Figure \ref{total_heat}.

\begin{figure}[h!]
\centerline{\includegraphics[width=15cm]{total_heat.png}}
\caption{Raw Total Heat Data}
\label{total_heat}
\end{figure}

\subsection{Processing}
The data is in chronological order and each row represents a data point after $0.1$s has elapsed from the previous row. There are a total of $4200$ rows representing $7$ minutes of data. I wrote a Python script that averages the total heat based on a given ($\Delta t$) number of data points. 
\subsection{Mapping}
The range of the average total heat is then divided into a given number of regions ($n$). These regions are then mapped to a given sequence of numbers ($p$), with a given number of lower regions dropped ($b$).
\subsection{Chosen Values}
Using $\Delta t = 50$, $n = 9$, $p = 1, 2, 3, 4, 5, 6, 7, 8$ and $b = 1$, the resulting mapped data is shown in Figure \ref{mapped_data}.

\begin{figure}[h!]
\centerline{\includegraphics[width=18cm]{average_heat.png}}
\caption{Average Total Heat and Mapped Values}
\label{mapped_data}
\end{figure}

\section{Interpretation}
When the leading trailing zeroes are stripped, the resulting string of values is \\
$100000156668754457422111$. I further discard the $100000$ prefix. 
\subsection{Rules}
The resulting string is used to construct a melodic line using the following rules:
\begin{enumerate}
\item Numbers indicate intervallic values. For example, $5$ is a $5$th, $6$ is a $6$th so on.
\item Intervals can be Major, Minor, Perfect, Augmented or Diminished.
\item Non-repeated numbers translate into note values at most a half-note.
\item Repeated numbers translate into note values longer than a half-note. Alternatively, they can be treated as multiple non-repeated numbers.
\end{enumerate}
\subsection{Examples}
Figure \ref{melodic_line} shows an example of the melodic material derived, first played by the second violin at m. $22$. It is used as the second theme in the composition, which then undergoes various treatments.

\begin{figure}[h!]
\centerline{\includegraphics[width=15cm]{melodic_line.png}}
\caption{Second theme of Fire, Violin 2, mm. 22 - 28}
\label{melodic_line}
\end{figure}

\begin{figure}[h!]
\centerline{\includegraphics[width=8cm]{fragmentation.png}}
\caption{Second theme with fragmentation and diminution, Violin 1, mm. 102 - 104}
\label{melodic_line}
\end{figure}

\subsection{Violations}
The rules are not always enforced after the first time the theme appears. Rule $1$ is sometimes relaxed to allow for the same intervals in different octaves (eg. a $2$nd is equivalent to a $9$th), and Rule $3$ and $4$ both admit violations for harmonic purposes. Figure \ref{second_theme} shows a brief excerpt with violations highlighted. Red boxes indicate violation of Rule $1$, blue boxes violations of Rule $3$ and green boxes violations of Rule $4$.

\begin{figure}[h!]
\centerline{\includegraphics[width=18cm]{violations.png}}
\caption{Second thematic material in all string parts, mm. 28 - 41}
\label{second_theme}
\end{figure}

\pagebreak
\section{Meta}
Besides composing with the rules in mind, I also made reference to fire in other ways.
\subsection{Visual}
Growing up in Malaysia, I experienced haze as a result of Indonesia's forest wildfires yearly. I have made the noteheads spell the word Fire in my native tongues - English, Malay and Chinese at various points in the piece. Examples can be found in Figure \ref{fire_english}, Figure \ref{fire_malay} and Figure \ref{fire_chinese}. Of note, the opening intro is replayed in the coda, but with the Violin 1 quoting the second theme in full while still retaining the word visuals.

\begin{figure}[h!]
\centerline{\includegraphics[width=5cm]{fire_english.png}}
\caption{Fire in English, mm. 1 - 4}
\label{fire_english}
\end{figure}

\begin{figure}[h!]
\centerline{\includegraphics[width=2cm]{fire_malay.png}}
\caption{Fire in Malay, mm. 116 - 118}
\label{fire_malay}
\end{figure}

\begin{figure}[h!]
\centerline{\includegraphics[width=2cm]{fire_chinese.png}}
\caption{Fire in Chinese,  m. 86}
\label{fire_chinese}
\end{figure}

\subsection{Aural}
The first theme starts with the upper strings in pizz. The texture builds up by making them play staccato, then bowed, then consecutive down bows, intending to emulate a fire starting from a spark that eventually goes out of control. The idea of fire ``building up'' also gets used twice in mm. 102 - 111 right before the coda.

\subsection{Structural}
The development of the piece transitions through the keys F (minor), B (minor), D (major) and E (minor), whose letters constitute the ciphertext of the word FIRE using \href{https://en.wikipedia.org/wiki/Musical_cryptogram#French}{French musical cryptogram}.

\subsection{Inspiration}
I am at most a mediocre composer, but ``imitation is the sincerest form of flattery that mediocrity can pay to greatness''.
\begin{enumerate}
\item The chord progression of the opening four bars and the multiple stops of the first violin is inspired by Bach's Chaconne.
\item The key of G minor and the cello ostinato is inspired by Vivaldi's Summer.
\item The first theme is inspired by Rachmaninoff's Cello Sonata's second movement's first theme.
\item The motif played by the first violin at m. 34 is inspired by Bach's first Cello Suite.
\item The stretto passage in m. 102 is inspired by Chopin's Fourth Ballade.
\end{enumerate}

\end{document}
